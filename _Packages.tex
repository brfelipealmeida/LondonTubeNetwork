% ------------------ PACKAGES ------------------ 
% Packages add extra commands and features to your LaTeX document. 
% In here, some of the most common packages for a thesis document have been added 

% LaTeX's float package
\usepackage{float}

% LaTeX's color package
\usepackage{color}

% LaTeX's main math package
\usepackage{amsmath}

% LaTeX's Caption and subcaption packages
\usepackage[format=hang,font=normalsize,labelfont=bf,labelsep=colon,singlelinecheck=off]{caption}
\usepackage{subcaption}

% The graphicx package provides graphics support for adding pictures.
\usepackage{graphicx}

% Longtable allows you to write tables that continue to the next page.
\usepackage{longtable}

% The geometry packages defines the page layout (page dimensions, margins, etc)
\usepackage[a4 paper, top=25mm, bottom=25mm]{geometry}

% Defines the Font of the document, e.g. Arimo font (Check Fonts here: https://tug.org/FontCatalogue/)
\usepackage[sfdefault]{arimo}

% Font encoding
\usepackage[T1]{fontenc}

% This package allows the user to specify the input encoding
\usepackage[utf8]{inputenc}

% This package allows you to add empty pages
\usepackage{emptypage}

% Allows inputs to be imported from a directory
\usepackage{import}

% Provides control over the typography of the Table of Contents, List of Figures and List of Tables
\usepackage{tocloft}

% The setspace package controls the line spacing properties.
\usepackage{setspace}

% Allows the customization of Latex's title styles
\usepackage{titlesec}

% Allows the customization of Latex's table of contents title styles
\usepackage{titletoc}

% The package provides functions that offer alternative ways of implementing some LATEX kernel commands
\usepackage{etoolbox}

% Provides extensive facilities for constructing and controlling headers and footers
\usepackage{fancyhdr} 

% Typographical extensions, namely character protrusion, font expansion, adjustment 
%of interword spacing and additional kerning
\usepackage{microtype}

% Manages hyperlinks 
\usepackage[colorlinks=true,linkcolor=black,urlcolor=blue]{hyperref}

% Generates PDF bookmarks
\usepackage{bookmark}

% Add color to Tables
\usepackage[table,xcdraw]{xcolor}

% Use these two packages together -- they define symbols
%  for e.g. units that you can use in both text and math mode.
\usepackage{gensymb}
\usepackage{textcomp}

% Bibliography package
\usepackage[backend=biber,style=nature]{biblatex}
\addbibresource{references.bib} % Add the .bib file that contains the references

% The natbib package allows book-type citations (e.g. (Saucer et al., 1993))
% More details are here: http://merkel.zoneo.net/Latex/natbib.php
%\usepackage[round]{natbib}

% The linenumbers command adds line numbers next to your text 
% That can be useful for discussing edits in drafts.
%\usepackage[modulo]{lineno}
%\linenumbers 


% This package provides an easy way to input latin sample text (for the template only)
\usepackage{blindtext}


\usepackage{booktabs}